\documentclass[../praca.tex]{subfiles}
\graphicspath{{\subfix{obrazky/}}}

\begin{document}

Stručný úvod do problematiky - dôvod, prečo ste sa ako autor rozhodli vypracovať
prácu na danú tému. Stanovuje cieľ práce, jej poslanie a~presné vymedzenie
problému, ktorým sa práca zaoberá. Používajú sa kratšie vety, nie zložité
súvetia. Má byť stručný a~výstižný a má prezentovať nasledujúci obsah práce.
Odporúčaný rozsah je jedna až jeden a pol strany. 

Aj keď je Úvod hneď na začiatku práce, obvykle sa píše až po jej dokončení. 

\subsection{Ešte pár všeobecných odporúčaní odo mňa pre vás, milí gvozáci}

Dajte si pozor na správne písanie interpunkcie - medzeru píšeme ZA interpunkčným
znamienkom, teda bodkou, čiarkou, otáznikom, nie pred nimi.

Prosím neprepisujte texty z~iných zdrojov bez citácie, v~texte musí byť jasné,
čo sú vaše myšlienky a~čo pred vami už niekto povedal. Môže to vyzerať napr.
takto:
% TODO
Bielkoviny v~ľudskom tele majú nezastupiteľnú úlohu pri tvorbe... (Mareš,
2010).

% TODO
Iný spôsob citácie: Ako uvádza Mareš (2010), bielkoviny v~ľudskom tele...

% TODO
V zozname použitej literatúry potom treba uviesť presný bibliografický odkaz na
túto literatúru - na poslednej strane tohto dokumentu nájdete ukážky, ako to
treba uviesť.

Pri výbere literatúry dbajte na jej aktuálnosť - nemali by ste citovať veľmi
staré knihy či webové zdroje, rozsah povedzme posledných 10 rokov je v norme.
(Samozrejme, sú knihy, ktoré sú vo svojom obore niečo ako Biblia, uznávané a
vážené, takže take môžete smelo citovať, aj keď sú staršieho dátumu vzniku.)

Pri písaní vašej práce by ste mali dokázať, že ste v~problematike dobre
zorientovaní, takže by ste mali rozhodne preštudovať viacero článkov, kníh,
stránok k~téme. (Vysoko odporúčam pohľadať aj literatúru v~anglickom jazyku
a~aspoň trošku čo-to prelúskať, ak vám nerobí angličtina veľké problémy, napr.
cez Google Scholar. Nie je to opäť nutné, no získali by ste tým pravdepodobne
lepší prehľad o problematike).
% TODO
Všetky zdroje, ktoré použijete v práci, treba mať
zapísané vzadu v už spomínanom zozname literatúry.

Práca nesmie byť celá len súhrnom cudzích myšlienok, k~téme, ktorej sa venujete,
by ste mali niečo vlastné dodať, zistiť, doplniť, vytvoriť...

Pri písaní používame prvú osobu množného čísla. Aj v~prípade, že ste jediný
autor práce, používate pri písaní osobu MY, teda autorský plurál.

Celkový rozsah práce 15 - 25 strán (od časti Úvod po Zoznam literatúry vrátane),
tento rozsah je nutné dodržať, priveľa alebo primálo strán môže znamenať
diskvalifikáciu vašej práce zo súťaže.

Držím vám palce a teším sa na prvé verzie vašich prác :). JP

\end{document}
