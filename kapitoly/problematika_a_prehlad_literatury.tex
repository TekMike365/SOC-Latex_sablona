\documentclass[../praca.tex]{subfiles}
\graphicspath{{\subfix{obrazky/}}}

\begin{document}

Sú to teoretické východiská, teoretická analýza problematiky. Táto teoretická
časť čitateľa stručne informuje o poznatkoch, ktoré boli v danej oblasti už
publikované.

Každú publikáciu, z ktorej využijeme informácie pri písaní Problematiky a
prehľadu literatúry, je potrebné citovať. \textbf{Odporúčaný rozsah tejto časti
    práce je tretina predkladanej práce.}

\subsection{Názov podkapitoly}

Text podkapitoly

\begin{figure}[ht]
    \centering
    \includegraphics[width=4cm]{obrazok.png}
    \caption{Príklad obrázka v texte}
    \label{obr:priklad}
\end{figure}

Príklad vloženej tabuľky:

\begin{table}[ht]
    \centering
    \caption{príklad tabuľky v texte}
    \label{tab:priklad}

    \vspace{6pt}

    \begin{tabularx}{0.8\textwidth}{|X|X|}
        \hline
        údaj & hodnota \\ \hline
        A    & 25      \\ \hline
        B    & 38      \\ \hline
    \end{tabularx}
\end{table}

\end{document}
